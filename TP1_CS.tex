\documentclass[french]{article}

\usepackage[utf8]{inputenc}
\usepackage[T1]{fontenc}
\usepackage{babel}
\usepackage{amsfonts}
\usepackage{amsmath}
\usepackage{amssymb}

\title{TP1 Calcul Scientifique}
\author{B Deporte}
\date{Novembre 2020}

\begin{document}

\maketitle

\section{Problème posé}

On considère l'équation de diffusion 1D (avec $ a>0 $) :

\begin{align}
\partial_{t}u(x,t)-\partial_{x}^{2}u(x,t)+au(x,t)=0,\hspace{2mm}(x,t)\in]0,1[\times[0,T]
\end{align}

avec les conditions aux limites :

\begin{align}
\partial_{x}u(x,t)&=0, \hspace{2mm}, x=0, t\in[0,T] \\
u(x,t)&=0, \hspace{2mm}, x=1, t\in[0,T] \\
u(x,0)&=\mathrm{cos}(\frac{\pi}{2}x), \hspace{2mm} x\in]0,1[, t=0
\end{align}

\section{Solution explicite}

Par la méthode de séparation des variables - on pose $ u(x,t)=\varphi(x)\psi(t) $

L'équation principale devient : $ \varphi(x)\psi'(t)-\varphi''(x)\psi(t)+a\varphi(x)\psi(t)=0 $

Et donc pour $ \phi(x)\psi(t) \neq 0 $ :

\[
\frac{\psi'(t)}{\psi(t)}-\frac{\varphi''(x)}{\varphi(x)}+a=0
\]

Les conditions aux limites permettent d'écrire :

\begin{align}
\varphi'(0)&=0 \\
\varphi(1)&=0 \\
\varphi(x)\psi(0)&=\mathrm{cos}(\frac{\pi}{2}x), \hspace{2mm} x\in]0,1[
\end{align}

On a donc : $ \frac{\psi'(t)}{\psi(t)}=\lambda \in \mathbb{R} $, et $ \frac{\varphi''(x)}{\varphi(x)}=\lambda+a $.

L'équation en $ \psi $ se résoud en $ \psi(t)=\psi(0)\exp(\lambda t) $

La solution de $ \frac{\varphi''(x)}{\varphi(x)}=\lambda+a $ dépend des racines de l'équation caractéristique $ r^{2}=\lambda+a $.

\paragraph{1er cas : $\lambda + a > 0$}

Alors $ \varphi(x)=A\exp(\sqrt{\lambda+a}x)+B\exp(-\sqrt{\lambda+a}x) $, avec $ A,B \in \mathbb{R} $.

Les conditions aux limites s'écrivent :
\begin{align}
\varphi'(0)&=A \sqrt{\lambda+a} - B \sqrt{\lambda+a}=0 \\
\varphi(1)&= A\exp(\sqrt{\lambda+a})+B\exp(-\sqrt{\lambda+a})=0 
\end{align}

D'où : $ A = B = 0 $ : pas de solution non nulle.

\paragraph{2e cas : $\lambda + a = 0$}

Alors $ \varphi''(x)=0 $, donc $ \varphi(x)=Ax+B $, et $ A=B=0 $ aussi avec les conditions aux limites. Pas de solution non nulle.

\paragraph{3e cas : $\lambda + a < 0$}

Les racines de l'équation caractéristique sont complexes, et : $ \varphi(x)=A\mathrm{cos}(\sqrt{-(\lambda+a)}x)+B\mathrm{sin}(-\sqrt{-(\lambda+a)}x) $

Les conditions aux limites donnent : $ \varphi'(0)=B=0 $, et $ \varphi(1)=A \mathrm{cos}(\sqrt{-(\lambda+a}))=0 $. 

Une solution non nulle $ A\neq 0 $ implique $ \sqrt{-(\lambda+a)} = \frac{\pi}{2}+k\pi $ avec $ k \in \mathbb{Z} $

Donc $ \varphi(x)=A\mathrm{cos(\frac{\pi}{2}+k\pi)x} $.

La dernière condition aux limites s'écrit alors : $ \psi(0)A \mathrm{cos(\frac{\pi}{2}+k\pi)x} = \mathrm{cos}(\frac{\pi}{2}x)$, ce qui donne : $ A\psi(0)=1, k=0 $

Au final, $ \varphi(x)\psi(t) = \mathrm{cos}(\frac{\pi}{2}x)\exp(\lambda t) $ avec $ -\lambda - a = \frac{\pi^{2}}{4} $, donc :

\[
u(x,t)=\mathrm{cos}(\frac{\pi}{2}x).\exp(-(a+\frac{\pi^{2}}{4})t)
\]

\section{Méthode des différences finies}

On considère un maillage uniforme espace-temps de $ [0,1] \times [0,T] $, avec $ \delta x = 1 / J $ et $ \delta t = T / N $.

Un point de la grille est donné par $ (x_{j},t_{n}) = (j.\delta x, n.\delta t) $ avec $ j \in (0,J) $ et $ n \in (0,N) $

On utilise les développements de Taylor à l'ordre 2 et à l'ordre 3 :

\begin{align}
v(x_{j+1}) &= v(x_{j}) + v'(x_{j}).\delta x + \frac{v''(x_{j})}{2!}.\delta x^{2} + O(\delta x^{3}) \\
v(x_{j-1}) &= v(x_{j}) - v'(x_{j}).\delta x + \frac{v''(x_{j})}{2!}.\delta x^{2} + O(\delta x^{3}) \\
v(x_{j+1}) &= v(x_{j}) + v'(x_{j}).\delta x + \frac{v''(x_{j})}{2!}.\delta x^{2} + \frac{v'''(x_{j+1})}{3!} + O(\delta x^{4}) \\
v(x_{j-1}) &= v(x_{j}) - v'(x_{j}).\delta x + \frac{v''(x_{j})}{2!}.\delta x^{2} - \frac{v'''(x_{j+1})}{3!} + O(\delta x^{4}) \\
\end{align}

pour approximer les dérivées première et seconde en espace :

\begin{align}
\partial_{x}v(x_{j})=\frac{v(x_{j+1})-v(x_{j-1})}{2\delta x}+O(\delta x^{2})
\end{align}

\begin{align}
\partial_{x}^{2}v(x_{j})=\frac{v(x_{j+1})-2v(x_{j})+v(x_{j-1})}{\delta x^{2}}+O(\delta x^{2})
\end{align}

On prend par contre l'approximation d'ordre 1 en temps, à savoir :

\[
\partial_{t}v(t_{n})= \frac{v(t_{n+1})-v(t_{n})}{\delta t} + O(\delta t)
\]

\subsection{Discrétisation complète avec $\theta$-schéma}

\subsubsection{Schéma explicite}

On pose $ u(x_{j},t_{n}) = u_{j}^{n} $.

L'équation (1) donne le schéma explicite (terme courant $ j \geq 1 $) :

\[
\frac{u_{j}^{n+1}-u_{j}^{n}}{\delta t}-\frac{u_{j+1}^{n}-2u_{j}^{n}+u_{j-1}^{n}}{\delta x^{2}}+au_{j}^{n}=0+O(\delta t)+O(\delta x^{2})
\]

Soit :

\begin{align}
\frac{1}{\delta t}u_{j}^{n+1}=\frac{1}{\delta t}u_{j}^{n}+
\frac{1}{\delta x^{2}}\left(u_{j+1}^{n}-2u_{j}^{n}+u_{j-1}^{n} \right)-au_{j}^{n}
\end{align}

Pour le terme en $ j=0 $, cad $ x=0 $, on utilise la condition aux limites (2), qui donne, en introduisant un terme fictif $ u_{-1}^{n} $ :

\[
\frac{u_{1}^{n}-u_{-1}^{n}}{2 \delta x} = 0
\]

donc $ u_{1}^{n} = u_{-1}^{n} $

et donc, pour $ j=0 $, l'équation (17) devient :

\[
\frac{1}{\delta t}u_{0}^{n+1}=\frac{1}{\delta t}u_{0}^{n}+
\frac{2}{\delta x^{2}}\left(u_{1}^{n}-u_{0}^{n} \right)-au_{0}^{n}
\]

La condition aux limites (3) donne $ u_{J}^{n}=0 $, on va donc poser simplement :

\[
U^{n}=(u_{0}^{n}, u_{1}^{n}, ... ,u_{J-1}^{n})^{T}
\]

On a alors :

\begin{align}
\frac{1}{\delta t}U^{n+1}=
\begin{pmatrix}
u_{0}^{n+1} \\ u_{1}^{n+1} \\ \hdots \\ u_{J-1}^{n+1}
\end{pmatrix} =
\frac{1}{\delta t}U^{n}
- \frac{1}{\delta x^{2}}
\begin{pmatrix}
2+a\delta x^{2} & -2 & \hdots \\
-1 & 2+a\delta x^{2} & -1 & \hdots \\
\ddots \\
\hdots & \hdots & -1 & 2+a\delta x^{2} 
\end{pmatrix}
\begin{pmatrix}
u_{0}^{n} \\ u_{1}^{n} \\ \hdots \\ u_{J-1}^{n}
\end{pmatrix}
\end{align}

Soit :

\begin{align}
\frac{1}{\delta t}\left(U^{n+1}-U^{n}\right) + AU^{n}=0
\end{align}

avec :

\begin{align}
A = \frac{1}{\delta x^{2}}
\begin{pmatrix}
2+a\delta x^{2} & -2 & \hdots \\
-1 & 2+a\delta x^{2} & -1 & \hdots \\
\ddots \\
\hdots & \hdots & -1 & 2+a\delta x^{2} 
\end{pmatrix}
\end{align}

\subsubsection{Schéma implicite et $\theta $ schéma}

Le schéma implicite se déduit du schéma explicite (19) :

\[
\frac{1}{\delta t}\left(U^{n+1}-U^{n}\right) + AU^{n+1}=0
\]

Et le theta-schéma est combinaison linéaire des deux schémas explicite (x $1-\theta$ ) et implicite (x $\theta$) :

\begin{align}
\frac{1}{\delta t}\left(U^{n+1}-U^{n}\right) + 
A\left((1-\theta)U^{n}+\theta U^{n+1}\right)=0
\end{align}

pour $ 0 \leq \theta \leq 1 $

\subsubsection{Condition stabilité du $\theta$-schéma}

On rappelle la condition de stabilité du $ \theta $-schéma, alias condition de stabilité de Von Neumann pour l'équation (1) :

\begin{align}
\mathrm{soit}\hspace{2mm}\theta &\geq 1/2 \\
\mathrm{si}\hspace{2mm}\theta \in [0,1/2[ : \frac{\delta t}{\delta x^{2}} &\leq \frac{1}{2(1-2\theta)}
\end{align}

\subsection{Mise en oeuvre $\theta$-schéma : voir NB Python}

\section{Méthode Elements Finis}

\subsection{Formulation variationnelle}

On va classiquement multiplier l'équation (1) par une fonction $ v : x \mapsto v(x) $ "suffisamment régulière" (on précisera les conditions plus loin), et intégrer en espace sur $ [0,1] $

On a donc :

\[
\int_{0}^{1}\partial_{t}u(x,t)v(x)dx - 
\int_{0}^{1}\partial_{x}^{2}u(x,t)v(x)dx + 
a\int_{0}^{1}u(x,t)v(x)dx = 0
\]

On intègre par parties la deuxième intégrale, et on suppose pour l'instant que l'on peut intervertir la dérivation par rapport au temps et l'intégration en espace dans la première intégrale (ce qui nécessite de vérifier par la suite quelques hypothèses) :

\[
\partial_{t}\int_{0}^{1}u(x,t)v(x)dx - 
\left(
\partial_{x}u(x,t)v(x)\mid_{0}^{1}-
\int_{0}^{1}\partial_{x}u(x,t)\partial_{x}v(x)dx 
\right) + 
a\int_{0}^{1}u(x,t)v(x)dx = 0
\]

On se rappelle que $\partial_{x}u(0,t) = 0$, et on impose $ v(1)=0 $, ce qui permet d'avoir au final :

\begin{align}
\int_{0}^{1}\partial_{t}u(x,t)v(x)dx +
\int_{0}^{1}\partial_{x}u(x,t)\partial_{x}v(x)dx +
a\int_{0}^{1}u(x,t)v(x)dx = 0
\end{align}

Pour que toutes les intégrales aient un sens, (et avec Cauchy-Schwartz), on peut donc imposer que $ v $ et $ u(.,t) $, pour tout $ t \in [0,T] $, soient dans $ L^{2} $, dérivables et de dérivée dans $ L^{2} $, avec également $ v(1) = 0 $. On a par les CLs $ u(1,t)=0 $, donc $ v $ et $ u(.,t) $ sont dans le même espace.

Concernant l'intervertion entre l'intégrale et la dérivation par rapport au temps : on a $ x \mapsto u(x,t) $ intégrable sur $ [0,1] $, $ t \mapsto u(x,t) $ dérivable pour tout $ t $. 

Restera à vérifier que $ \exists g:x\mapsto g(x) $ intégrable sur $ [0,1] $ avec $ \forall t \in [0,T], |\partial_{t}u(x,t)| \leq g(x) $. (ce qui est le cas avec la formulation explicite de la solution)

\subsection{Théorie du Schéma numérique}

\subsubsection{Formulation}

On considère le maillage en espace $ [x_{0}=0, x_{1},...,x_{i},...,x_{J-1},x_{J}=1] $.

Le degré 2 des éléments finis choisis, nécessite de faire intervenir le milieu de chaque $ [x_{i},x_{i+1}] $ : on définit ainsi les noeuds $ \tilde{x}_{j} $ avec $ 0 \leq j \leq 2J $, et $ \tilde{x}_{2j} = x_{i} $ et $ \tilde{x}_{2j+1} = (x_{j}+x_{j+1}) / 2 $. En clair : $ [x_{i},x_{i+1}] = [\tilde{x}_{2i}, \tilde{x}_{2i+1}, \tilde{x}_{2i+2}]$

Les 3 éléments de Lagrange P2 pour le segment $ [0,1] $ de noeuds $ [0, 1/2, 1] $ sont :

\begin{align}
\psi_{1} &= 2(x-1/2)(x-1) \\
\psi_{2} &= -4x(x-1) \\
\psi_{3} &= 2(x-1/2)x
\end{align}

On fait donc intervenir les $ \varphi_{j} $ avec $ 0 \leq j \leq 2J $.

On pose $ \tilde{u}_{j}^{n} = u(\tilde{x}_{j},n\delta t) $ la valeur de $ u $ au noeud $ \tilde{x}_{j} $ au temps $ t^{n} = n \delta t $ (avec donc $ 0 \leq j \leq 2J $)

La solution approchée de $ u $ est alors (sachant que $ \tilde{u}^{n}_{2J} = 0 $ par les conditions aux limites ):

\[
u^{n}=\sum_{j=0}^{2J-1}\tilde{u}^{n}_{j}\varphi_{j}
\]

On écrit alors la formulation variationnelle (24) en prenant $ v = \varphi_{i} $ pour $ 0 \leq i \leq 2J-1 $ :

\begin{align}
\forall i \in [0,2J-1], 
\partial_{t}\int_{0}^{1}\sum_{j=0}^{2J-1}\tilde{u}_{j}^{n}\varphi_{j}\varphi_{i}
+ \int_{0}^{1}\sum_{j=0}^{2J-1}\tilde{u}_{j}^{n}\varphi_{j}'\varphi_{i}'
+ a \int_{0}^{1}\sum_{j=0}^{2J-1}\tilde{u}_{j}^{n}\varphi_{j}\varphi_{i} =0
\end{align}

Avec au premier ordre : 

\begin{align}
\partial_{t}u &= \sum_{j=0}^{2J-1}\partial_{t}\tilde{u}_{j}(t)\varphi_{j} \\
&= \sum_{j=0}^{2J-1}\left( \frac{\tilde{u}_{j}^{n+1}-\tilde{u}_{j}^{n}}{\delta t}\right)
+O(\delta t)
\end{align}

On pose la matrice de masse $ M = (m_{ij}) $ de taille 2J x 2J avec

\[
m_{ij}=\int_{0}^{1}\varphi_{i}\varphi_{j} 
\]

Et la matrice de raideur $ K = (k_{ij}) $ de taille 2J x 2J avec

\[
k_{ij}=\int_{0}^{1}\varphi_{i}'\varphi_{j}' 
\]

L'équation (26) s'écrit, avec $ \tilde{U}^{n}=(\tilde{u}_{0}^{n}, \tilde{u}_{1}^{n}, ..., \tilde{u}_{2J-1}^{n}) $ :

\[
\forall i \in [0,J-1], 
\sum_{j=0}^{2J-1}\left( \frac{\tilde{u}_{j}^{n+1}-\tilde{u}_{j}^{n}}{\delta t}\right)m_{ij}+
\sum_{j=0}^{2J-1}k_{ij}\tilde{u}_{j}^{n}+
a\sum_{j}^{2J-1}m_{ij}\tilde{u}^{n}_{j}=0
\]

Sous forme matricielle :

\[
\frac{M}{\delta t}(\tilde{U}^{n+1}-\tilde{U}^{n})+K\tilde{U}^{n}+aM\tilde{U}^{n}=0
\]

\begin{align}
\frac{M}{\delta t}\tilde{U}^{n+1}=
\left(
\frac{M}{\delta t}-K-aM
\right)\tilde{U}^{n}
\end{align}

\subsubsection{Matrices de masse et de rigidité}

Les matrices élémentaires de masse et de rigidité sont de taille 3x3, puisque $ \int \varphi_{i}\varphi_{j} = 0 $ et $ \int \varphi_{i}'\varphi_{j}' = 0 $ dès que $ |j-i| \geq 3 $

Elles sont données, pour l'intervalle $ [x_{k},x_{k+1}] $ par :

\begin{align}
M_{elem}= \frac{x_{k}-x_{k-1}}{30}
\left(
\begin{matrix}
4 & 2 & -1 \\
2 & 16 & 2 \\
-1 & 2 & 4
\end{matrix}
\right)
\end{align}

\begin{align}
K_{elem}= \frac{1}{x_{k}-x_{k-1}}\frac{1}{3}
\left(
\begin{matrix}
7 & -8 & 1 \\
-8 & 16 & -8 \\
1 & -8 & 7
\end{matrix}
\right)
\end{align}

L'assemblage des matrices complètes se fait en ajoutant les matrices 3x3 élémentaires le long de la diagonale principale (cf code).


\subsubsection{Schémas numériques}

\paragraph{Schéma Explicite}

\[
\frac{M}{\delta t}(\tilde{U}^{n+1}-\tilde{U}^{n})+K\tilde{U}^{n}+aM\tilde{U}^{n}=0
\]

\begin{align}
\frac{M}{\delta t}\tilde{U}^{n+1}=
\left(
\frac{M}{\delta t}-K-aM
\right)\tilde{U}^{n}
\end{align}

\paragraph{Schéma Implicite}

\[
\frac{M}{\delta t}(\tilde{U}^{n+1}-\tilde{U}^{n})+K\tilde{U}^{n+1}+aM\tilde{U}^{n+1}=0
\]

\begin{align}
\left(
\frac{M}{\delta t}+K+aM
\right)\tilde{U}^{n+1}=
\frac{M}{\delta t}
\tilde{U}^{n}
\end{align}

\paragraph{$\theta$ schéma}

Donné par (34) x $ 1 - \theta $ + (35) x $ \theta $

\[
\frac{M}{\delta t}(\tilde{U}^{n+1}-\tilde{U}^{n})
+(1-\theta)*(K+aM)\tilde{U}^{n}
+\theta*(K+aM)\tilde{U}^{n+1}=0
\]

\[
\left(
\frac{M}{\delta t}+\theta(K+aM)
\right)\tilde{U}^{n+1}-
\left(
\frac{M}{\delta t}-(K+aM)(1-\theta)\right)\tilde{U}^{n}=0
\]

Soit finalement :

\begin{align}
\left(
\frac{M}{\delta t}+\theta(K+aM)
\right)
\tilde{U}^{n+1} = 
\left(
\frac{M}{\delta t}-(1-\theta)(K+aM)
\right)
\tilde{U}^{n}
\end{align}








\end{document}